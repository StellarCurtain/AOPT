\begin{document}
\title{\textbf{高级优化 (2024 秋季)\\
作业}}
\author{学号, 姓名, 邮箱}
\maketitle

\begin{tcolorbox}
\textbf{评分规则}: 共包含 5 道题目(270 分)。题目~\ref{sec:APG}(70 分)要求必做。从剩下的 4 道题目中选择任意 3 道(每题 50 分)完成。最终得分评估有以下两种选项:
\begin{enumerate}
  \item (\textbf{推荐}) 如果完成 4 道题目(题目~\ref{sec:APG} + 任意选择的 3 道,总计 220 分),当得分至少为 200 分时,可以获得满分(200 分)。
  \item 如果完成 4 道题目(总计 220 分)\emph{并完成剩余的一道题目(50 分)}:
    \begin{enumerate}
      \item 如果在已选的 4 道题目中未能达到 200 分,则按照第一种情况计算得分。
      \item 如果总分为 $(245+X)$ 分($X \ge 0$),最终得分为 $(200+X)$ 分。
    \end{enumerate}
\end{enumerate}

\begin{center}
  \textbf{注意: 请清楚标明选择的题目编号。} 

  {\color{red}\mychoice{x,x,x}}  \\  
  \texttt{\% 用选择的编号替换 x,x,x(例如 2,3,4,5)}
\end{center}
\end{tcolorbox}



\newpage
\section{[70 分] APG 分析与实现}
\label{sec:APG}
考虑以下无约束的复合优化问题:
\begin{align*}
  \min_{\x \in \mathbb{R}^d} F(\x) \triangleq f(\x) + h(\x),
\end{align*}
其中 $f(\cdot)$ 和 $h(\cdot)$ 均为凸函数。函数 $f(\cdot)$ 是 $L$-光滑的,而 $h(\cdot)$ 不是。

近端梯度法 (Proximal Gradient, PG) 的更新公式为 $\x_{t+1} = \P^h_L(\x_t) \triangleq \prox_{\frac{1}{L} h} (\x_t - \frac{1}{L} \nabla f(\x_t))$,其中 $\prox$ 是近端映射。PG 的收敛率为 $\O(1/T)$,但这一速率是次优的,与光滑优化中梯度下降法的次优性类似。改进复合优化问题的收敛率的一种自然方法是扩展 Nesterov 加速梯度下降法 (Accelerated Gradient Descent, AGD),由此得到了加速近端梯度法 (Accelerated Proximal Gradient, APG) 算法:
\begin{tcolorbox}[top=-1pt]
  \begin{align*}
    \x_{t+1} = \P^h_L(\y_t),\quad \y_{t+1} = \x_{t+1} + \beta_t(\x_{t+1} - \x_t),
  \end{align*}
  其中 $\beta_t > 0$ 是一个随时间变化的“动量”项 ($\x_{t+1} - \x_t$) 的权重。
\end{tcolorbox}

\begin{enumerate}
  \item[(1)] \textbf{[15 分]} \underline{设计} $\beta_t$ 并 \underline{证明} APG 的收敛率:
  \begin{align*}
    F(\x_T) - F(\x^\star) \le \O\left(\frac{1}{T^2}\right). 
  \end{align*}
  \item[(2)] \textbf{[10 分]} 对于 Lipschitz 连续的凸函数,我们知道梯度下降法 (GD) 的收敛率为 $\O(1/\sqrt{T})$。鉴于 APG 方法的优异性能,我们可能会想,是否可以“修改”APG 以获得 Lipschitz 连续凸函数的更快收敛率?
  
  具体而言,对于凸且 Lipschitz 连续的优化问题 $\min_{\x \in \R^d} h(\x)$,我们可以将其重写为复合优化问题 $\min_{\x \in \R^d} f(\x) + h(\x)$,其中 $f(\x) = 0$ 是一个凸且 $0$-光滑的函数,满足 APG 的要求。因此,APG 的结果似乎直接暗示了 Lipschitz 连续凸函数 $h(\x)$ 的 $\O(1/T^2)$ 收敛率。即使 $L=0$ 可能会在近端映射中引发问题,我们也可以添加一个小的 $\epsilon$ 进行修正。这种想法是否正确?给出你的答案,并 \underline{简要说明原因}。
  \item[(3)] \textbf{[5 分]} 为了更好地理解 APG,考虑以下实际应用:从视频中进行背景建模。假设给定一个数据矩阵 $D\in\mathbb{R}^{m\times d}$,我们期望将其分解为
  \begin{align*}
    D = L_0 + S_0,
  \end{align*}
  其中 $L_0 \in \R^{m\times d}$ 是低秩矩阵,$S_0 \in \R^{m \times d}$ 是稀疏矩阵。例如,如果数据矩阵 $D$ 表示监控视频的帧序列,背景变化 $L_0$ 可以建模为低秩结构,因为帧间存在相关性,而前景运动物体 $S_0$ 可以建模为稀疏分量。
  
  为实现这一目标,我们将其形式化为以下优化问题:
  \begin{align}
    \label{eq:objective}
    \min_{L, S\in\mathbb{R}^{m\times d}} \frac{1}{2}\norm{D - L - S}_\text{F}^2 + \mu \norm{L}_* + \lambda \norm{S}_1,
  \end{align}
  其中 $\mu,\lambda>0$ 是超参数,$\norm{A}_*=\sum_{i}\sigma_i(A)=\text{tr}(\sqrt{A^\top A})$ 表示核范数,用于施加低秩约束,$\norm{A}_1=\sum_{ij}|A_{ij}|$ 表示 $\ell_1$-范数,用于施加稀疏性约束。
  
  为求解问题~\eqref{eq:objective},我们可以将其转换为复合优化问题,其中优化变量为 $\mathbf{X} \triangleq (X^L, X^S)\in\mathbb{R}^{m\times d}\times \mathbb{R}^{m\times d}$,对应的复合函数为 $f(\mathbf{X})=\frac{1}{2}\norm{D - X^L - X^S}_\text{F}^2$ 和 $h(\mathbf{X})=\mu \norm{X^L}_* + \lambda \norm{X^S}_1$。此时优化问题变为
  \begin{align*}
    \min_{\mathbf{X}\in\mathbb{R}^{m\times d}\times \mathbb{R}^{m\times d}} f(\mathbf{X}) + h(\mathbf{X}).
  \end{align*}
  注意,$f(\cdot)$ 和 $h(\cdot)$ 都是凸函数,且 $f(\cdot)$ 是关于 $\norm{\cdot}$ 范数的 $L_f$-光滑函数(即 $\norm{\mathbf{X}}\triangleq\sqrt{\norm{X^L}_{\text{F}}^2+\norm{X^S}_{\text{F}}^2}$)。
  
  \underline{计算}该问题中 $f(\cdot)$ 的光滑参数 $L_f$。

  \item[(4)] \textbf{[40 分]} 我们可以进一步使用 APG 算法求解视频背景建模问题。请 \underline{实现} PG 和 APG 算法,\underline{比较} PG 和 APG 的损失曲线,并在此处 \underline{附上图表}。详细说明可见文件 \texttt{AOpt-Lab1/AOpt-Lab1.ipynb} 中的 Jupyter Notebook。请确保将完成的 Notebook 导出为 HTML 文件。请确保所有输出可见,并且 \underline{\emph{提交 HTML 文件作为作业的一部分}}。
\end{enumerate}
\begin{solution}
请在此填写你的答案。(可用中文或英文书写)
~\\
~\\
~\\
\end{solution}
